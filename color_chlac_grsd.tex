\documentclass[conference]{sty/IEEEtran}
\usepackage{times}
\usepackage{wrapfig}
\usepackage{tweaklist}
\usepackage{xspace}
\usepackage{graphicx}
\usepackage{subfigure}
\usepackage{tabularx}
\usepackage{amsmath}
\usepackage{amssymb}
\usepackage{url}


% numbers option provides compact numerical references in the text. 
\usepackage[numbers]{natbib}
\usepackage{multicol}
\usepackage[bookmarks=true]{hyperref}

\pdfinfo{
   /Author (Homer Simpson)
   /Title  (Robots: Our new overlords)
   /CreationDate (D:20101201120000)
   /Subject (Robots)
   /Keywords (Robots;Overlords)
}

\begin{document}

% paper title
\title{Classification of Objects of Daily Use Using Combined Color CHLAC and Global Radius-based Surface Descriptors}

% You will get a Paper-ID when submitting a pdf file to the conference system
\author{Author Names Omitted for Anonymous Review. Paper-ID [add your ID here]}

%\author{\authorblockN{Michael Shell}
%\authorblockA{School of Electrical and\\Computer Engineering\\
%Georgia Institute of Technology\\
%Atlanta, Georgia 30332--0250\\
%Email: mshell@ece.gatech.edu}
%\and
%\authorblockN{Homer Simpson}
%\authorblockA{Twentieth Century Fox\\
%Springfield, USA\\
%Email: homer@thesimpsons.com}
%\and
%\authorblockN{James Kirk\\ and Montgomery Scott}
%\authorblockA{Starfleet Academy\\
%San Francisco, California 96678-2391\\
%Telephone: (800) 555--1212\\
%Fax: (888) 555--1212}}


% avoiding spaces at the end of the author lines is not a problem with
% conference papers because we don't use \thanks or \IEEEmembership


% for over three affiliations, or if they all won't fit within the width
% of the page, use this alternative format:
% 
%\author{\authorblockN{Michael Shell\authorrefmark{1},
%Homer Simpson\authorrefmark{2},
%James Kirk\authorrefmark{3}, 
%Montgomery Scott\authorrefmark{3} and
%Eldon Tyrell\authorrefmark{4}}
%\authorblockA{\authorrefmark{1}School of Electrical and Computer Engineering\\
%Georgia Institute of Technology,
%Atlanta, Georgia 30332--0250\\ Email: mshell@ece.gatech.edu}
%\authorblockA{\authorrefmark{2}Twentieth Century Fox, Springfield, USA\\
%Email: homer@thesimpsons.com}
%\authorblockA{\authorrefmark{3}Starfleet Academy, San Francisco, California 96678-2391\\
%Telephone: (800) 555--1212, Fax: (888) 555--1212}
%\authorblockA{\authorrefmark{4}Tyrell Inc., 123 Replicant Street, Los Angeles, California 90210--4321}}


\maketitle

\begin{abstract}
The abstract goes here.
\end{abstract}

\IEEEpeerreviewmaketitle

\section{Introduction}

\section{Related Work}
\begin{itemize}
\item VFH
\item GRSD (Humanoids10GRSD)\cite{kalman1960new} 
\item Color CHLAC (Asako)
\end{itemize}

\section{System Overview}


\section{Feature Estimation}

\subsection{Color CHLAC}
\subsection{GRSD}
\begin{figure}[htb!]
  \begin{center}
    \includegraphics[width=.4\columnwidth]{figures/grsd/book.png}
\hfill
    \includegraphics[width=.48\columnwidth]{figures/grsd/book_global.png} \\
\hfill
    \includegraphics[width=.3\columnwidth]{figures/grsd/mug.png}
\hfill
    \includegraphics[width=.48\columnwidth]{figures/grsd/mug_global.png}
\caption{Example of RSD classes and GRSD plots for a big flat box (i.e. book, upper row) and a short cylinder (i.e. mug, bottom row).
The histogram bin values are scaled between -1 and 1 according to the
training data, and the colors represent the following local surfaces:
red - sharp edge (or noise), yellow - plane, green - cylinder, light blue -
sphere (not present), and dark blue - rim (i.e. boundary, transition between surfaces).
\emph{Best viewed in color.}
}
    \label{fig:gfpfh}
  \end{center}
\vspace{-2ex}
\end{figure}

\section{Classification Methods}

\subsection{Linear Subspace Method}
\subsection{Support Vector Machine-based Classification}


\section{Results}

\subsection{Data Acquisition and Training}
\subsection{Online Testing/Object Recognition}


\section{Conclusions and Future Work} 
The conclusion goes here.

\section*{Acknowledgments}
CoTeSys
%% Use plainnat to work nicely with natbib. 

\bibliographystyle{plainnat}
\bibliography{references}

\end{document}


%TODO:
%Check the original template and see what the meant with the hyperlinks
